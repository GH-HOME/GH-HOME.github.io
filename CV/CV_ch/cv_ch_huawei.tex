\documentclass[UTF8]{ctexart}
%\documentclass[A4, 11pt, UTF8]{ctexart}
\usepackage[T1]{fontenc}
%-------------------------
% Resume in Latex
% Author : Jake Gutierrez
% Based off of: https://github.com/sb2nov/resume
% License : MIT
%------------------------

%\documentclass[letterpaper,11pt]{article}

\usepackage{latexsym}
\usepackage[empty]{fullpage}
\usepackage{titlesec}
\usepackage{marvosym}
\usepackage[usenames,dvipsnames]{color}
\usepackage{verbatim}
\usepackage{enumitem}
\usepackage[hidelinks]{hyperref}
\usepackage{fancyhdr}
\usepackage[english]{babel}
\usepackage{tabularx}
%\usepackage{zh_CN-Adobefonts_external} 
\usepackage{linespacing_fix}
\input{glyphtounicode}
\usepackage{marvosym}
\usepackage{tikz}


%----------FONT OPTIONS----------
% sans-serif
% \usepackage[sfdefault]{FiraSans}
% \usepackage[sfdefault]{roboto}
% \usepackage[sfdefault]{noto-sans}
% \usepackage[default]{sourcesanspro}

% serif
% \usepackage{CormorantGaramond}
% \usepackage{charter}


\pagestyle{fancy}
\fancyhf{} % clear all header and footer fields
\fancyfoot{}
\renewcommand{\headrulewidth}{0pt}
\renewcommand{\footrulewidth}{0pt}

% Adjust margins
\addtolength{\oddsidemargin}{-0.5in}
\addtolength{\evensidemargin}{-0.5in}
\addtolength{\textwidth}{1in}
\addtolength{\topmargin}{-.5in}
\addtolength{\textheight}{1.0in}

\urlstyle{same}

\raggedbottom
\raggedright
\setlength{\tabcolsep}{0in}

% Sections formatting
\titleformat{\section}{
	\vspace{-4pt}\scshape\raggedright\large
}{}{0em}{}[\color{black}\titlerule \vspace{-5pt}]

% Ensure that generate pdf is machine readable/ATS parsable
\pdfgentounicode=1

%-------------------------
% Custom commands
\newcommand{\resumeItem}[1]{
	\item\small{
		{#1 \vspace{-2pt}}
	}
}

\newcommand{\resumeSubheading}[4]{
	\vspace{-2pt}\item
	\begin{tabular*}{0.97\textwidth}[t]{l@{\extracolsep{\fill}}r}
		#1 & #2 \\
		\textit{\small#3} & \textit{\small #4} \\
	\end{tabular*}\vspace{-7pt}
}

\newcommand{\resumeSubSubheading}[2]{
	\item
	\begin{tabular*}{0.97\textwidth}{l@{\extracolsep{\fill}}r}
		\textit{\small#1} & \textit{\small #2} \\
	\end{tabular*}\vspace{-7pt}
}

\newcommand{\resumeProjectHeading}[2]{
	\item
	\begin{tabular*}{0.97\textwidth}{l@{\extracolsep{\fill}}r}
		 #1 & \small #2 \\
	\end{tabular*}\vspace{-7pt}
}

\newcommand{\resumeSubItem}[1]{\resumeItem{#1}\vspace{-4pt}}

\renewcommand\labelitemii{$\vcenter{\hbox{\tiny$\bullet$}}$}

\newcommand{\resumeSubHeadingListStart}{\begin{itemize}[leftmargin=0.15in, label={}]}
	\newcommand{\resumeSubHeadingListEnd}{\end{itemize}}
\newcommand{\resumeItemListStart}{\begin{itemize}}
	\newcommand{\resumeItemListEnd}{\end{itemize}\vspace{-5pt}}

%-------------------------------------------
%%%%%%  RESUME STARTS HERE  %%%%%%%%%%%%%%%%%%%%%%%%%%%%


\begin{document}
	\begin{minipage}[c]{0.9\textwidth}
	\begin{center}
		\textbf{\Huge \scshape  \textit{郭$\,$亨}} \\ \vspace{14pt}
		(086)18582521993 $|$ \href{mailto:heng.guo@ist.osaka-u.ac.jp}{\underline{heng.guo@ist.osaka-u.ac.jp}} $|$ 
		\href{https://scholar.google.com/citations?user=HKu6gF4AAAAJ&hl=zh-CN}{\underline{Google Scholar}} $|$ 
	\end{center}
	\end{minipage}
	\hspace{-2em}
	\begin{minipage}[r]{0.05\textwidth}
		\centering
		\raisebox{0.6em}{\includegraphics[width = 1.5cm]{homepage}}\\
		\vspace{-1em}
		\href{https://gh-home.github.io/}{\underline{个人主页}}
	\end{minipage}

%	\begin{center}
%	\textbf{\Huge \scshape  \textit{郭$\,$亨}} \\ \vspace{10pt}
%	(086)18582521993 $|$ \href{mailto:heng.guo@ist.osaka-u.ac.jp}{\underline{heng.guo@ist.osaka-u.ac.jp}} $|$ 
%	\href{https://scholar.google.com/citations?user=HKu6gF4AAAAJ&hl=zh-CN}{\underline{Google Scholar}} $|$ 
%	\href{https://gh-home.github.io/}{\underline{个人主页}}
%	\end{center}
	
	\vspace{-0.5em}
	%-----------EDUCATION-----------
	\section{\textit{\textbf{教育背景}}}
	\resumeSubHeadingListStart
	\resumeSubheading
	{电子科技大学}{成都, 中国}
	{电子信息工程,学士}{\rm  2011年9月 -- 2015年7月}
		\vspace{0.3em}
	\resumeSubheading
	{电子科技大学}{成都, 中国}
	{信号与信息处理,硕士 $|$ 导师: 曾兵,\href{http://www.liushuaicheng.org/}{刘帅成}}{\rm 2015年9月 -- 2018年7月}
		\vspace{0.3em}
	\resumeSubheading
	{大阪大学}{大阪, 日本}
	{多媒体信息处理, 博士 $|$ 导师:\href{http://cvl.ist.osaka-u.ac.jp/en/member/matsushita/}{Yasuyuki Matsushita},\href{https://cs.pku.edu.cn/info/1073/1888.htm}{施柏鑫}}{\rm  2018年10月 -- 今}
	\resumeSubHeadingListEnd
	
	
	%-----------PROJECTS-----------
	\section{\textit{\textbf{获奖荣誉}}}
	\resumeSubHeadingListStart
	\resumeProjectHeading
	{电子科技大学优秀毕业论文 (Top 3\%)}{2018年6月}
	\resumeProjectHeading
	{电子科技大学优秀硕士毕业生 (Top 6\%)}{2018年6月}
	\resumeProjectHeading
	{研究生国家奖学金 (Top 2\%))}{2017年10月}
	\resumeProjectHeading
	{电子科技大学学术奖学金 (Top 10\%)}{2015 \& 2016 \& 2017}
	\resumeProjectHeading
	{电子科技大学优秀毕业生 (Top 7\%)}{2015年9月}
	\resumeProjectHeading
	{全国信息安全竞赛一等奖}{2014年7月}
	\resumeProjectHeading
	{四川省电子设计竞赛二等奖}{2013年9月}
	\resumeProjectHeading
	{人民奖学金 (Top 15\%)}{2011 \& 2012 \& 2013}
%	%   \resumeProjectHeading
%	%   {Outstanding Individual in Social Practice}{Sep. 2012}
	\resumeSubHeadingListEnd
	
	
	\section{\textit{\textbf{项目经历}}}
	\begin{enumerate}[label={[\arabic*]}]
	\item 自然未标定光源下的高精度三维重建 (TPAMI 2021)
	\begin{itemize}
		\item 将光度立体算法从暗室拍摄条件拓展到自然光
		\item 提出等效平行光模型来近似局部区域的自然光照
	\end{itemize}
	\item 基于光度立体的高精度动态三维重建 (CVPR 2021, IJCV 2021)
	\begin{itemize}
		\item 使用多光谱相机和多频段光源实现基于单帧多光谱图像的形状恢复
		\item 将多光谱光度立体算法从病态问题转换为可解问题
	\end{itemize}
	\item 非均匀散射近点光源下的光度立体算法 (MIRU 最佳学生论文)
	\begin{itemize}
		\item 实现在实际非均匀散射LED照射情况下基于光度立体的三维重建
		\item 提出一种更具表达能力的光源散射模型
	\end{itemize}
	\item 视频防抖拼接算法 (TIP 2016)
	\begin{itemize}
		\item 提出一种基于网格的特征检测算法
		\item 提出基于网格的相机路径优化算法从而实现视频拼接和视频稳像
	\end{itemize}
	\item 基于双目视频的稳像算法 (TCI 2018, ICIP 2016)
	\begin{itemize}
		\item 在平稳双目视频的同时保持原有的双目视频帧间视差
		\item 提出一种基于网格的形变算法JDSW
	\end{itemize}
	\item 基于Android的实时视频稳像算法 (电子科技大学优秀硕士毕业论文)
	\begin{itemize}
		\item 提出一种实时的相机路径优化算法
		\item 将优化算法集成在Android APP中实现在线视频稳像,输出视频与输入仅有0.5s时延
	\end{itemize}
	\item 基于非朗伯物体的动态三维重建
	\item 高精度三维重建中的边缘保持算法
	\end{enumerate}
	
	\section{\textit{\textbf{发表论文}}}
	%\renewcommand\labelenumi{(\theenumi)}
	\begin{enumerate}[label={[\arabic*]}]
		\item Guo Heng, et al. `Patch-based Uncalibrated Photometric Stereo under Natural Illumination" 
		IEEE Transactions on Pattern Analysis and Machine Intelligence. (\textbf{TPAMI 2021}) [SCI 1区, 影响因子: 16.39]. 
		\item Guo Heng, et al. ``Multispectral Photometric Stereo for Spatially-varying Spectral Reflectances: A Well-posed Problem?" IEEE Conference on Computer Vision and Pattern Recognition. (\textbf{CVPR 2021}) [CCF A类]. 
		\item Guo Heng, et al. ``Self-calibrating Near-light Photometric Stereo under Anisotropic Light Emission." Meeting on Image Recognition and Understanding (\textbf{MIRU 2020 最佳学生论文}).  
		\item Guo Heng, et al. ``Joint Video Stitching and Stabilization from Moving Cameras." IEEE Transactions on Image Processing (\textbf{TIP 2016}) [SCI 1区, 影响因子: 6.79].
		\item Guo Heng, et al. ``View-consistent Meshflow for Stereoscopic Video Stabilization." IEEE Transactions on Computational Imaging (\textbf{TCI 2018}) [SCI 2区, 影响因子: 3.49].
		\item Guo Heng, et al. ``Joint Bundled Camera Paths for Stereoscopic Video Stabilization." IEEE International Conference on Image Processing (\textbf{ICIP 2016 Oral}) [CCF C类].
	\end{enumerate}
	\begin{itemize}[label={[*]}]
		\item Guo Heng, et al. ``Multispectral Photometric Stereo for Spatially-varying Spectral Reflectances" International Journal of Computer Vision. (\textbf{IJCV} minor revision) [SCI 1区, 影响因子: 7.41]. 
		\item Guo Heng, et al. ``Parametric Near-light Photometric Stereo" IEEE Conference on Computer Vision and Pattern Recognition. (\textbf{CVPR 2022} under review). 
		\item Guo Heng, et al. ``NeuralMPS: Multispectral Photometric Stereo for Non-lambertian Spectral Reflectance" IEEE Conference on Computer Vision and Pattern Recognition. (\textbf{CVPR 2022} under review). 
	\end{itemize}


		%-----------EXPERIENCE-----------
	\section{\textit{\textbf{工作经历}}}
	\resumeSubHeadingListStart
	\resumeSubheading
	{大阪大学}{大阪, 日本}
	{研究助理 \quad$|$  导师:\href{http://cvl.ist.osaka-u.ac.jp/en/member/matsushita/}{Yasuyuki Matsushita}}{\rm  2018年10月 -- 今}
	\vspace{0.3em}
	\resumeSubheading
	{OPPO日本研究院}{横滨, 日本}
	{算法实习生  $|$ 导师:\href{https://www.linkedin.com/in/ahmedboudissa/?originalSubdomain=jp}{Ahmed Boudissa}}{\rm  2021年8月 -- 2021年10月}
	\resumeItemListStart
	\resumeItem{基于Dual-Pixel的深度估计算法.}
	\resumeItemListEnd
	\vspace{0.3em}
	% -----------Multiple Positions Heading-----------
	\resumeSubheading
	{奇虎360人工智能研究院}{北京, 中国}
	{算法实习生 $|$ 导师: \href{https://www.cs.sfu.ca/~pingtan/}{谭平}}{\rm  2016年6月 -- 2017年1月}
	\resumeItemListStart
	\resumeItem{手机端实时视频稳像算法及APP研发.}
	\resumeItemListEnd
	%	\resumeItemListStart
	%	\resumeItem{Propose a real-time video stabilization algorithm for user-captured videos with 0.5s delay.}
	%	\resumeItem{Implement an android application for real-time video stabilization.}
	%	\resumeItemListEnd
	
	\resumeSubHeadingListEnd
	
	%
	%-----------PROGRAMMING SKILLS-----------
	%\section{Skills}
	% \begin{itemize}[leftmargin=0.15in, label={}]
		%    \small{\item{
				%     \textbf{Program language}{: Java, Python, C/C++} \\
				%     \textbf{Certificacts}{: TOEFL (95), CET-6 (527)} \\
				%    }}
		% \end{itemize}
	
	
	%-------------------------------------------
\end{document}
